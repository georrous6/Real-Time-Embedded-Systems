\documentclass{beamer}
\usepackage{graphicx}
\usepackage{booktabs}
\usepackage{hyperref}
\usetheme{Madrid}

\title[CHROMOSOME RTE]{CHROMOSOME: A Plug \& Play-Capable Run-Time Environment for Embedded Real-Time Systems}
\author{Buckl et al., 2014}
\institute{fortiss GmbH \& Technische Universität München}
\date{May 2025}

\begin{document}

% TITLE
\begin{frame}
  \titlepage
\end{frame}

% SLIDE 1
\begin{frame}{Motivation: Why Adaptive Embedded Systems?}
  \begin{itemize}
    \item Traditional embedded systems rely on static configuration — no updates post-deployment.
    \item New-generation systems must support:
    \begin{itemize}
      \item Dynamic integration of new functionality (e.g., OTA updates).
      \item Fault-tolerant reconfiguration in distributed topologies.
    \end{itemize}
    \item Example: A modern car may have 100+ ECUs with different software life cycles.
    \item Challenge: Combine plug \& play capabilities with hard real-time guarantees.
    \item Existing middleware lacks the timing predictability required by safety-critical systems.
  \end{itemize}
\end{frame}

% SLIDE 2
\begin{frame}{CHROMOSOME (XME): What Is It?}
  \begin{itemize}
    \item A modular run-time environment designed for embedded real-time systems.
    \item Built to support plug \& play at both software and network levels.
    \item Cross-platform and open-source — usable across industrial and automotive domains.
    \item Resource-aware: Uses lookup tables to ensure that added components fit system limits.
    \item Components are dynamically loaded only if sufficient memory, CPU time, and bandwidth exist.
    \item Compatible with both time-triggered and event-driven execution models.
  \end{itemize}
\end{frame}

% SLIDE 3
\begin{frame}{Core Concepts of XME}
  \begin{itemize}
    \item \textbf{Requirements-centric design}: Application developers specify constraints like WCET, latency, safety level.
    \item \textbf{Data-centric communication}: No hard-coded connections; communication defined by topic types and attributes.
    \item \textbf{Plug \& play orchestration}: New nodes/components can register and integrate safely at run-time.
    \item \textbf{Shadow configuration}: Ensures system consistency during reconfiguration — existing tasks are not interrupted.
    \item \textbf{Topic dictionaries}: Promote interoperability across independently developed modules.
    \item \textbf{Modularity}: Different runtime components (e.g., schedulers) can be selected per platform.
  \end{itemize}
\end{frame}

% SLIDE 4
\begin{frame}{XME Architecture Overview}
  \begin{itemize}
    \item The XME ecosystem is a federation of runtime nodes.
    \item Each node includes core services: Execution Manager, Broker, Data Handler, PAL.
    \item Nodes communicate via directed, type-safe publish-subscribe channels.
    \item The architecture separates application logic from low-level hardware using PAL.
    \item Reconfiguration is handled by plug-phase components: Login and Plug \& Play Managers.
    \item Adaptivity features are only enabled if the node includes plug-phase services.
  \end{itemize}
\end{frame}

% SLIDE 5
\begin{frame}{Play-Phase Components}
  \begin{itemize}
    \item \textbf{Execution Manager (EM)}:
    \begin{itemize}
      \item Invokes application components according to their priority and WCET.
      \item Monitors for timing violations; reports if a task exceeds limits.
    \end{itemize}
    \item \textbf{Broker}:
    \begin{itemize}
      \item Watches data subscriptions and enables components only when inputs are ready.
    \end{itemize}
    \item \textbf{Data Handler}:
    \begin{itemize}
      \item Provides a consistent data interface to components.
      \item Ensures safe, bounded-latency access to shared data.
    \end{itemize}
    \item These ensure deterministic behavior during the “play” (runtime) phase.
  \end{itemize}
\end{frame}

% SLIDE 6
\begin{frame}{Plug-Phase Components}
  \begin{itemize}
    \item \textbf{Login Client/Manager}:
    \begin{itemize}
      \item Detects new nodes and establishes secure initial communication.
    \end{itemize}
    \item \textbf{Plug \& Play Client/Manager}:
    \begin{itemize}
      \item Exchange manifests that describe required topics, timing, safety, etc.
      \item Validates whether system resources can support the new component.
      \item Coordinates with Network and Resource Configurators to compute updated tables.
    \end{itemize}
    \item Final reconfiguration applied atomically once shadow configuration passes.
  \end{itemize}
\end{frame}

% SLIDE 7
\begin{frame}{Communication via Topics and Dictionaries}
  \begin{itemize}
    \item XME replaces traditional service calls with publish/subscribe over semantic “topics.”
    \item Each topic includes:
    \begin{itemize}
      \item Data type (e.g., sensor\_reading, vehicle\_position).
      \item Qualifiers: min/max range, units, criticality level, precision.
    \end{itemize}
    \item Topic matching ensures syntactic and semantic compatibility between components.
    \item Enables components developed independently to interoperate safely.
  \end{itemize}
\end{frame}

% SLIDE 8
\begin{frame}{Waypoints and Platform Abstraction Layer (PAL)}
  \textbf{Waypoints:}
  \begin{itemize}
    \item Lightweight modules for data transformation and verification.
    \item Can include marshaling (endianness), redundancy, CRCs, encryption, etc.
    \item Configured dynamically per data route based on system requirements.
  \end{itemize}
  \textbf{Platform Abstraction Layer (PAL):}
  \begin{itemize}
    \item Provides uniform access to hardware (GPIO, timers, interrupts).
    \item Makes applications portable across OSes like PikeOS, Linux, or even bare metal.
  \end{itemize}
\end{frame}

% SLIDE 9
\begin{frame}{Tool Support: XMT Modeling Tool}
  \begin{itemize}
    \item Eclipse-based graphical environment for designing XME systems.
    \item Developers can:
    \begin{itemize}
      \item Declare component behavior, topics, and constraints.
      \item Validate system resource allocations and check for timing violations.
      \item Generate code and configuration files for deployment.
    \end{itemize}
    \item Supports both static and dynamic configuration workflows.
  \end{itemize}
\end{frame}

% SLIDE 10
\begin{frame}{Use Case 1: RACE (Automotive)}
  \begin{itemize}
    \item Objective: enable adaptive automotive architecture with hard real-time support.
    \item Deployed using PikeOS partitions — each app runs in a secure container.
    \item XME runs in a master partition; manages partition communication and coordination.
    \item Real-time scheduling configured based on WCET data and component criticality.
    \item Allows runtime replacement of software modules without rebooting the ECU.
  \end{itemize}
\end{frame}

% SLIDE 11
\begin{frame}{Use Case 2: AutoPnP (Industrial Automation)}
  \begin{itemize}
    \item Targeted for factories needing reconfiguration without halting production.
    \item Hardware modules are detected at runtime; software activated accordingly.
    \item Event-triggered model handles inputs like sensor events or operator changes.
    \item Uses ROS for interoperability; XME adds plug \& play and timing control.
    \item Demonstrates flexibility in industrial robotics and assembly systems.
  \end{itemize}
\end{frame}

% SLIDE 12
\begin{frame}{Comparison with Other Frameworks}
  \begin{tabular}{l|l|l}
    \textbf{System} & \textbf{Real-Time Support} & \textbf{Adaptivity} \\
    \hline
    ROS & No & High \\
    DDS & Partial (QoS-based) & Medium \\
    FRESCOR & Contract-based & High \\
    \textbf{CHROMOSOME} & \textbf{WCET + Static Sched.} & \textbf{High + Modular}
  \end{tabular}
  \vspace{1em}
  \smallskip
  XME uniquely blends static real-time safety with dynamic reconfigurability.
\end{frame}

% SLIDE 13
\begin{frame}{Future Directions}
  \begin{itemize}
    \item Full end-to-end timing analysis across multi-node ecosystems.
    \item Binary deployment of application components, not just nodes.
    \item Security-aware middleware: role- and topic-based access control.
    \item Health monitoring and self-healing orchestration.
  \end{itemize}
\end{frame}

% SLIDE 14
\begin{frame}{Conclusion}
  \begin{itemize}
    \item XME enables real-time embedded systems to adapt safely and modularly.
    \item Combines model-driven design, static analysis, and runtime orchestration.
    \item Plug \& play becomes practical for cyber-physical systems (CPS).
    \item Publicly available — suitable for research and industrial applications.
  \end{itemize}
\end{frame}

% SLIDE 15
\begin{frame}{Thank You!}
  \centering
  \Huge Questions?
\end{frame}

\end{document}

